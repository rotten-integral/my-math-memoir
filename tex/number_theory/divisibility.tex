\section{Divisibility}
\subsection{Problems}
\begin{problem}{1}{\href{https://artofproblemsolving.com/community/q2h3107356p34893836}{IMO Shortlist N2, 2022}}
	Find all positive integers $n>2$ such that
$$ n! \mid \prod_{ p<q\le n, p,q \, \text{primes}} (p+q).$$
\end{problem}

\begin{problem}{2}{\href{https://artofproblemsolving.com/community/c6h3239264p34615369}{India P4, 2024}}
	Let $p$ be an odd prime and $a,b,c$ be integers so that the integers$$a^{2023}+b^{2023},\quad b^{2024}+c^{2024},\quad a^{2025}+c^{2025}$$are divisible by $p$.Prove that $p$ divides each of $a,b,c$.
\end{problem}

\begin{problem}{3}{\href{https://artofproblemsolving.com/community/c7h2709003p34748857}{Simon Marais A2, 2021}}
	Define the sequence of integers $a_1, a_2, a_3, \ldots$ by $a_1 = 1$, and
\[ a_{n+1} = \left(n+1-\gcd(a_n,n) \right) \times a_n \]for all integers $n \ge 1$. Prove that $\frac{a_{n+1}}{a_n}=n$ if and only if $n$ is prime or $n=1$. Here $\gcd(s,t)$ denotes the greatest common divisor of $s$ and $t$.
\end{problem}

\begin{problem}{4}{\href{https://web.evanchen.cc/handouts/CRT/CRT.pdf}{CRT-Evan Chen's Handout}}
	Let $n$ be a positive integer. Determine, in terms of $n$, the number of $x\in\{1,2, \ldots, n\}$ for which $x^2\equiv x\pmod{n}$.
\end{problem}

\newpage
\subsection{Solutions}
\begin{problem}{1}{\href{https://artofproblemsolving.com/community/q2h3107356p34893836}{IMO Shortlist N2, 2022}}
	Find all positive integers $n>2$ such that
$$ n! \mid \prod_{ p<q\le n, p,q \, \text{primes}} (p+q).$$
	\begin{solution} We claim that $n=7$ is the only solution, easy to check why it works.\\
	\indent Now we will show that no other value works. Clearly any $n<7$ and $7< n\le 11$ does not work and let us assume some $n>11$ works. Let $p_{\text{max}}$ be the largest prime less than $n$. By the given given condition, it is evident that distinct prime $p<q\le n$ exist that satisfy,
$$p+q\equiv 0\pmod{p_{\text{max}}}.$$But, note as $p<q\le p_{\text{max}}$ it must be that $p+q=p_{\text{max}}$ and by parity constraint, it is necessary that $p=2$. Similarly there exist primes $p'<q'\le n$ such that $p'+q'\equiv 0\pmod{p_{\text{max}}-2}$. By comparing sizes, it is clear that either
$$p'+q'=p_{\text{max}}-2\quad\text{or}\quad p'+q'=2p_{\text{max}}-4.$$We shall deal with the latter case first. Both $p'$ and $q'$ cannot be less than $p_{\text{max}}$ else we have a contradiction, so this forces $p_{\text{max}}-4$ to also be a prime. Recall that there does not exist a prime $p$ such that both $p+2$ and $p+4$ are primes unless $p=3$ (as a hint consider modulo $3$). So this case is ruled out and the former case is obvious as $p$ must be $2$ and $p_{\text{max}}-4$ is a prime which is a contradiction.$\qedwhite$

	\begin{remark}[title=Comment.$\hspace{1mm}$]
	The problem does not justify its positioning in N2 by any means.
	\end{remark}
	\end{solution}
\end{problem}
	
\begin{problem}{2}{\href{https://artofproblemsolving.com/community/c6h3239264p34615369}{India P4, 2024}}
		Let $p$ be an odd prime and $a,b,c$ be integers so that the integers$$a^{2023}+b^{2023},\quad b^{2024}+c^{2024},\quad a^{2025}+c^{2025}$$are divisible by $p$.Prove that $p$ divides each of $a,b,c$.
	\begin{solution} Showing that any one of $a,b,c$ is divisible by $p$ would suffice. Assume all of them are invertible on $\mathbb{Z}_p$. As $b^{2024}\equiv -ba^{2023}$ we may say,
$$c^{2024}+b^{2024}\equiv c^{2024}-ba^{2023}\quad\text{and}\quad c^{2025}\equiv cba^{2023}.$$We then have $bca^{2023}+a^{2025}\equiv 0$. Since $a$ is invertible we are left with the key result,
$$a^2\equiv -bc.$$Using the above fact, $a(-bc)^{1012}\equiv-c^{2025}$ which deduces to $ab^{1012}\equiv -c^{1013}$ and as $a^{2025}\equiv cb^{2024}$ we can easily show that $b^{1012}\equiv ac^{1011}$. Now we are left with,
$$ab^{1012}\equiv a^2c^{1011}\equiv -c^{1013}\quad\implies\quad a^2\equiv -c^2.$$But we had already seen $a^2\equiv -bc$ and is absurd, contradicting our assumption, hence none of them are invertible as desired.$\qedwhite$
	\end{solution}
\end{problem}

\begin{problem}{3}{\href{https://artofproblemsolving.com/community/c7h2709003p34748857}{Simon Marais A2, 2021}}
	Define the sequence of integers $a_1, a_2, a_3, \ldots$ by $a_1 = 1$, and
\[ a_{n+1} = \left(n+1-\gcd(a_n,n) \right) \times a_n \]for all integers $n \ge 1$. Prove that $\frac{a_{n+1}}{a_n}=n$ if and only if $n$ is prime or $n=1$. Here $\gcd(s,t)$ denotes the greatest common divisor of $s$ and $t$.
	\begin{solution}Let us write down few terms just to build some intuition,
	\begin{center}\begin{tabular}{c|c c c c c c c c c}
$a_n$ & $a_1$ & $a_2$ & $a_3$ & $a_4$ & $a_5$ & $a_6$ & $a_7$ & $a_8$ & $a_9$\\
\hline
$n$ & $1$ & $1$ & $2$ & $2\cdot 3$ & $2\cdot 3^2$ & $2\cdot 3^2\cdot 5$ & $2\cdot 3^2\cdot 5$ &$2\cdot 3^2\cdot 5\cdot 7$ & $2\cdot 3^2\cdot 5\cdot 7^2$\\
	\end{tabular}\end{center}
	\indent Notice that every prime less than $n$ divides $a_n$ for $n\le 12$ at least and more precisely no prime bigger than $N-1$ divides $a_N$ as well. We ask the question, if this holds for any $n$.

	\begin{numclaim}{1}
		A prime $p$ divides $a_n$ if and only if $p<n$.
	\end{numclaim}
	\begin{proof} This simply follows from an inductive argument (base case holds-refer the table). Suppose this property holds for some natural $N$ then we have,
$$a_{N+1}=\underbrace{(N+1-\gcd(N, a_N))}_{\text{less than $N+1$}}\times a_N $$By hypothesis it is true that every prime less than $N$ divides $a_N$ and are the only ones. As $N+1-\gcd(N, a_N)< N+1$, our claim holds!
	\end{proof}
	\noindent To finish off we just need to observe that $\gcd(n, a_n)=1$ if and only if $n$ is a prime which is obvious from our claim and we are done.$\qedwhite$
	\end{solution}
\end{problem}

\begin{problem}{4}{\href{https://web.evanchen.cc/handouts/CRT/CRT.pdf}{CRT-Evan Chen's Handout}}
	Let $n$ be a positive integer. Determine, in terms of $n$, the number of $x\in\{1,2, \ldots, n\}$ for which $x^2\equiv x\pmod{n}$.
	\begin{solution} The answer is $2^k$, where $k$ corresponds to the number of distinct prime divisors of $n$.\\
	Say $n=p_1^{e_1}p_2^{e_2}\cdots p_k^{e_k}$. We then have,
	\begin{align*}
		x^2 &\equiv x\pmod {p_1^{e_1}}\\
		x^2 &\equiv x\pmod {p_2^{e_2}}\\
		&\vdotswithin{\equiv}\\
		x^2 &\equiv x\pmod {p_k^{e_k}}.
	\end{align*}
	\indent As $\gcd(x, (x-1))=1$, each time we either have $p_i^{e_i}\lvert x$ or $p_i^{e_i}\lvert x-1$ (something like a binary tree). So, there are $2^k$ possible constructions for $x$ and hence the answer. $\qedwhite$
	\end{solution}
\end{problem}
