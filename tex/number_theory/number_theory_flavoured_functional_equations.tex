\section{Number Theory Flavoured Functional Equations}
\subsection{Problems}
\begin{problem}{1}{\href{https://artofproblemsolving.com/community/q2h2529938p34791441}{USAJMO P1, 2021}}
	Let $\mathbb{N}$ denote the set of positive integers. Find all functions $f : \mathbb{N} \rightarrow \mathbb{N}$ such that for positive integers $a$ and $b,$\[f(a^2 + b^2) = f(a)f(b) \text{ and } f(a^2) = f(a)^2.\]
\end{problem}

\begin{problem}{2}{\href{https://artofproblemsolving.com/community/c6h1854148p12519631}{APMO P1, 2019}}
	Let $\mathbb{N}$ be the set of positive integers. Determine all functions $f:\mathbb{N}\to\mathbb{N}$ such that $a^2+f(a)f(b)$ is divisible by $f(a)+b$ for all positive integers $a,b$.
\end{problem}

\begin{problem}{3}{\href{https://artofproblemsolving.com/community/c6h597243p3544096}{IMO Shortlist N1, 2013}}
	Let $\mathbb{Z} _{>0}$ be the set of positive integers. Find all functions $f: \mathbb{Z} _{>0}\rightarrow \mathbb{Z} _{>0}$ such that
\[ m^2 + f(n) \mid mf(m) +n \]
for all positive integers $m$ and $n$.
\end{problem}

\newpage
\subsection{Solutions}
\begin{problem}{1}{\href{https://artofproblemsolving.com/community/q2h2529938p34791441}{USAJMO P1, 2021}}
	Let $\mathbb{N}$ denote the set of positive integers. Find all functions $f : \mathbb{N} \rightarrow \mathbb{N}$ such that for positive integers $a$ and $b,$\[f(a^2 + b^2) = f(a)f(b) \text{ and } f(a^2) = f(a)^2.\]
	\begin{solution} We claim that $f\equiv 1$, easy to check why it works and we shall show that it is the only solution.\\
Let us evaluate $f$ at first few points to build some intuition. Putting $a=1$ in $f(a^2)=(f(a))^2$, we get $f(1)=1$ and put $a=1$ in $f(2a^2)=(f(a))^2$ to get $f(2)=1$. Repeating these steps multiple times, we arrive at,
$$f(1)=f(2)=f(3)=f(4)=f(5)=1.$$Computing $f(3)=1$ is slightly tricky, one might have to use the fact that $3^2+4^2=5^2$ (this gives us the hint that the problem might have something to do with Pythagorean triplets). So, we are somewhat convinced that $f\equiv 1$.
Let us setup a strong induction argument, assume $f(n)=1$ whenever $n\le N$ for some natural $N\ge 5$. We will show that $f(N+1)=1$. To do so we will use a famous result regarding \bluehighlight{Primitive Pythagorean Triplets}.
		\begin{theorem}{1}{Primitive Pythagorean Triplets} A triple of integers $(x,y,z)$ is a primitive Pythagorean triple if and only if $x=r^2-s^2$, $y=2rs$ and $z=r^2+s^2$, where $r,s$ are arbitrary integers of opposite parity $r>s>0$ and $\gcd(r,s)=1$.
	\end{theorem}
	For further details, refer section $5.3$ of An Introduction to the Theory of Numbers authored by Ivan Niven, Herbert S. Zuckerman and Hugh L. Montgomery.\\

	\begin{mycases}
		\item When $N+1$ is odd.\\
		In reference to theorem $1$, setting $s=r-1$ we have have $r^2-(r-1)^2=N+1$ for some $r\le N$. By our induction hypithesis, we know that,
		$$f((r^2-s^2)^2+(2rs)^2)=f(r^2-s^2)f(2rs)=f(r^2+s^2)=f(r)f(s)=1.$$So $f(r^2-(r-1)^2)=f(N+1)=1$ as desired.

		\item When $N+1$ is even.\\
		Similar in spirit as that of case $1$, there exists $r,s$ with $2rs=N+1$. Again by our induction hypothesis, we have,
		$$f((r^2-s^2)^2+(2rs)^2)=f(r^2-s^2)f(2rs)=f(r^2+s^2)=f(r)f(s)=1.$$Clearly, $f(2rs)=f(N+1)=1$.
	\end{mycases}

	\noindent This completes our induction argument, giving us $f\equiv 1$ as desired. $\qedwhite$
	\end{solution}
\end{problem}

\begin{problem}{2}{\href{https://artofproblemsolving.com/community/c6h1854148p12519631}{APMO P1, 2019}}
	Let $\mathbb{N}$ be the set of positive integers. Determine all functions $f:\mathbb{N}\to\mathbb{N}$ such that $a^2+f(a)f(b)$ is divisible by $f(a)+b$ for all positive integers $a,b$.
	\begin{solution} We claim that $f=\text{id}$ is the only solution and it's easy check the validity.\\
		\indent Now, we proceed to show that no other solution exists. The key idea in these kind of problems is to exploit divisibility constraints with the help of primes along with size arguments. And this intuition yields the following result.
		
		\begin{claim}
			Given any prime $p$, it follows that $f(p)=p$.
		\end{claim}
		\begin{proof}
			Let $p$ denote a prime. Now substituting $a$ and $b$ with $p$, we have,
			\[p^2+f(p)^2\equiv 0\pmod{p+f(p)}\quad\text{implies}\quad 2pf(p)\equiv 0\pmod{p+f(p)} .\]
			Now, we either have $\gcd(f(p), p)=p$ or $\gcd(f(p), p)=1$ and readily the latter is inferred as ridiculous. Which leaves us with the only possibility,
			\[2f(p)\equiv 0\pmod{1+\tfrac{f(p)}{p}}.\]
			Evidently, $1+\tfrac{f(p)}{p}$ shares a common factor with neither $f(p)$ nor $p$ yielding $f(p)=p$ as desired.
		\end{proof}
		
		\indent Consider a prime $p$. From the hypothesis, we have for any $a$, that $p+a\lvert p^2+pf(a)$ and by cancellation rule it follows that,
		\[p+f(a)+a-a\equiv 0\pmod {p+a}\quad\text{or}\quad f(a)-a\equiv 0\pmod{p+a}.\]
		But making $p$ sufficiently large contradicts the above unless $f(a)-a=0$ and we are done. $\qedwhite$
	\end{solution}
\end{problem}

\begin{problem}{3}{\href{https://artofproblemsolving.com/community/c6h597243p3544096}{IMO Shortlist N1, 2013}}
	Let $\mathbb{Z} _{>0}$ be the set of positive integers. Find all functions $f: \mathbb{Z} _{>0}\rightarrow \mathbb{Z} _{>0}$ such that
\[ m^2 + f(n) \mid mf(m) +n \]
for all positive integers $m$ and $n$.
	\begin{solution}
		We claim that $f=\mathrm{id}$ is the only solution and can be checked easily.\\
		Now, we proceed to show that this is the only solution. Begin by substituting $m$ with $f(n)$ by which we have,
		\[f(n)f\circ f(n)+n\equiv 0\pmod {f(n)^2+f(n)}.\]
		\indent And as $\gcd(f(n), f(n)+1)=1$, it follows that $f(n)\lvert n$ (we have a clear bound, $f(n)\le n$). This instantly yields $f(1)=1$. Now we finish off by by substituting $n$ with $1$ in the intitial condition, so,
		\[mf(m)+1 \equiv 0\pmod{m^2+1}.\]
		But this is only possible when $f(m)\ge m$. Simultaneously, recall that $f(m)\le m$! thus yielding us the required. $\qedwhite$
	\end{solution}
\end{problem}
	

	

