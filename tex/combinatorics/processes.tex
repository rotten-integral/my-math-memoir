\chapter{Processes}
\section{Problems}
\begin{problem}{1}{\href{https://artofproblemsolving.com/community/q1h3107334p34892180}{IMO Shortlist C1, 2022}} 
	A $\pm 1$-sequence is a sequence of $2022$ numbers $a_1, \ldots, a_{2022},$ each equal to either $+1$ or $-1$. Determine the largest $C$ so that, for any $\pm 1$-sequence, there exists an integer $k$ and indices $1 \le t_1 < \ldots < t_k \le 2022$ so that $t_{i+1} - t_i \le 2$ for all $i$, and$$\left| \sum_{i = 1}^{k} a_{t_i} \right| \ge C.$$
\end{problem}

\begin{problem}{2}{\href{https://artofproblemsolving.com/community/q1h3483100p34496847}{India P5, 2025}}
	Greedy goblin Griphook has a regular $2000$-gon, whose every vertex has a single coin. In a move, he chooses a vertex, removes one coin each from the two adjacent vertices, and adds one coin to the chosen vertex, keeping the remaining coin for himself. He can only make such a move if both adjacent vertices have at least one coin. Griphook stops only when he cannot make any more moves. What is the maximum and minimum number of coins he could have collected?
\end{problem}

\begin{problem}{3}{\href{https://artofproblemsolving.com/community/q1h2883211p34203276}{IMO P1, 2022}}
	The Bank of Oslo issues two types of coin: aluminum (denoted A) and bronze (denoted B). Marianne has $n$ aluminum coins and $n$ bronze coins arranged in a row in some arbitrary initial order. A chain is any subsequence of consecutive coins of the same type. Given a fixed positive integer $k \leq 2n$, Gilberty repeatedly performs the following operation: he identifies the longest chain containing the $k^{th}$ coin from the left and moves all coins in that chain to the left end of the row. For example, if $n=4$ and $k=4$, the process starting from the ordering $AABBBABA$ would be $AABBBABA \to BBBAAABA \to AAABBBBA \to BBBBAAAA \to ...$\\
	Find all pairs $(n,k)$ with $1 \leq k \leq 2n$ such that for every initial ordering, at some moment during the process, the leftmost $n$ coins will all be of the same type.
\end{problem}

\begin{problem}{4}{\href{https://artofproblemsolving.com/community/q1h1578516p32946994}{India P5, 2018}}
	There are $n\ge 3$ girls in a class sitting around a circular table, each having some apples with her. Every time the teacher notices a girl having more apples than both of her neighbours combined, the teacher takes away one apple from that girl and gives one apple each to her neighbours. Prove that, this process stops after a finite number of steps. (Assume that, the teacher has an abundant supply of apples.)
\end{problem}

\newpage
\section{Solutions}
\begin{problem}{1}{\href{https://artofproblemsolving.com/community/q1h3107334p34892180}{IMO Shortlist C1, 2022}} 
A $\pm 1$-sequence is a sequence of $2022$ numbers $a_1, \ldots, a_{2022},$ each equal to either $+1$ or $-1$. Determine the largest $C$ so that, for any $\pm 1$-sequence, there exists an integer $k$ and indices $1 \le t_1 < \ldots < t_k \le 2022$ so that $t_{i+1} - t_i \le 2$ for all $i$, and$$\left| \sum_{i = 1}^{k} a_{t_i} \right| \ge C.$$
	\begin{solution} We claim that the optimal value of $C$ is $506$.\\
Constructing a sequence that gives an upper bound of $506$ on $C$ is very natural. Consider the sequence,
$$\textcolor{blue}{1},\textcolor{blue}{1}, \textcolor{red}{-1}, \textcolor{red}{-1}, \textcolor{blue}{1},\textcolor{blue}{1}, \textcolor{red}{-1}, \textcolor{red}{-1},\ldots, \textcolor{blue}{1},\textcolor{blue}{1}, \textcolor{red}{-1}, \textcolor{red}{-1},\textcolor{blue}{1}, \textcolor{red}{-1}.$$There are $505$ copies of $1,1,-1, -1$ chunks with an anomaly at the end. Say we want to maximise the number of $1$s in the list, then we must have at least $507$ more $1$s than $-1$s and this is impossible as appending a $1$ in our list comes at the cost of appending a $-1$ unless we are yet to start.\\

	\noindent \textbf{\textcolor{darkolivegreen}{Establishing The Lower bound on $C$.}} We will show that no matter what $a_1,a_2,\ldots, a_{2022}$ are, we can always achieve
$$\left|\sum_{1\le i\le k}a_{t_i}\right|\ge 506 \qquad\text{with $t_{i+1}-t_i\le 2$,}$$for some $k$. Say there are at least $1011$ ones in the sequence and we want to include all of it in the list i.e, $a_{t_1}, a_{t_2}, \ldots, a_{t_k}.$ We do so according to the following algorithm,
	\begin{itemize}
		\item Begin from the smallest index that contains a $1$ and append it to the list.
		\item If the next nearest $1$ is at most two steps away, jump to that index and append it to the list.
		\item And if you don't find a $1$, then always jump to the right by two steps until you find a $1$ and of course appending the number you jumped on each time. Eventually if you encounter a $1$ which is at most two steps away, follow the above procedure.
	\end{itemize}
	\indent Aha! following these steps, the worst case is you would have collected at most $505$ (a jump of length $2$ is essentially avoiding a $-1$) number of $-1$s in the list and we are done. $\qedwhite$
	\end{solution}
\end{problem}

\begin{problem}{2}{\href{https://artofproblemsolving.com/community/q1h3483100p34496847}{India P5, 2025}}
	Greedy goblin Griphook has a regular $2000$-gon, whose every vertex has a single coin. In a move, he chooses a vertex, removes one coin each from the two adjacent vertices, and adds one coin to the chosen vertex, keeping the remaining coin for himself. He can only make such a move if both adjacent vertices have at least one coin. Griphook stops only when he cannot make any more moves. What is the maximum and minimum number of coins he could have collected?
	\begin{solution} We claim Griphook can pick a maximum of $1998$ coins and a minimum of $668$ coins.
We prove a more general proposition: that given any $n$-gon for $n\ge 3$, Griphook can pick at most $n-2$ coins and the minimizing case is a bit tricky given by the following formula,
	$$\text{Minimum number of coins Griphook must pick}=
	\begin{cases}
		\lceil n/3\rceil+1 &\text{if $n\equiv 2\pmod{3}$,}\\\\
 		\lceil n/3\rceil & \text{otherwise.}
	\end{cases}$$
	(this proposition is simply motivated by checking for small cases). Firstly, we shall deal with the maximizing case and we shall do that by constructing an algorithm for collecting $n-2$ coins and then arguing that it cannot get any better. For convenience we represent the number of coins on each vertex as a string with $i^{\text{th}}$ number representing the number of coins on the $i^{\text{th}}$ vertex. Consider picking the coins in the following manner, i.e, pick the next vertex each time starting from the second vertex,
	$$\textcolor{blue}{1}\underline{\textcolor{blue}{1}}\textcolor{blue}{1111}\ldots \textcolor{blue}{111}\to \textcolor{red}{0}\textcolor{green}{2}\underline{\textcolor{red}{0}}\textcolor{blue}{111}\ldots\textcolor{blue}{111}\to \textcolor{red}{0}\textcolor{blue}{11}\underline{\textcolor{red}{0}}\textcolor{blue}{11}\ldots \textcolor{blue}{111}\to \textcolor{red}{0}\textcolor{blue}{1}\textcolor{red}{0}\textcolor{blue}{1}\underline{\textcolor{red}{0}}\textcolor{blue}{1}\ldots \textcolor{blue}{111}\to \cdots\to \textcolor{red}{0}\textcolor{blue}{1}\textcolor{red}{0000}\ldots \textcolor{red}{0}\textcolor{blue}{1}\textcolor{red}{0}.$$As you may see there will be exactly two coins left that are two vertices away from each other at the end. Suppose we could do better, clearly we cannot pick all the $n$ coins, so we are left with the only possibility of picking $n-1$ coins and assume we could. By backtracking we have,
	$$\textcolor{red}{0}\textcolor{blue}{1}\textcolor{red}{0000}\ldots \textcolor{red}{000}\gets \textcolor{blue}{1}\textcolor{red}{0}\textcolor{blue}{1}\textcolor{red}{000}\ldots \textcolor{red}{000}\gets \textcolor{blue}{11}\textcolor{red}{0}\textcolor{blue}{1}\textcolor{red}{00}\ldots \textcolor{red}{000}\gets \textcolor{blue}{111}\textcolor{red}{0}\textcolor{blue}{1}\textcolor{red}{0}\ldots \textcolor{red}{000}\gets \cdots \gets \textcolor{blue}{111111}\ldots \textcolor{blue}{1}\textcolor{red}{0}\textcolor{blue}{1}.$$Note that the above sequence of moves is the only possibility (ahh ignore the cyclic permutations). But, the configuration we get at the end is never achievable hence $n-2$ is the best upper bound.\\
	\indent Now we shall deal with the minimizing case. Divide the string into substrings each of length $3$ and for Griphook to stop, it must be that each of them contain at least one $0$ at the extremities. Notice, each move can generate only one substring with at least a single $0$ at the extremity, hence the lower bound. When $n\equiv 2\pmod{3}$ there will always be a $\overline{\textcolor{green}{2}\textcolor{red}{0}\textcolor{green}{2}}$ string at the end and we are done. $\qedwhite$

	\end{solution}
\end{problem}

\begin{problem}{3}{\href{https://artofproblemsolving.com/community/q1h2883211p34203276}{IMO P1, 2022}}
	The Bank of Oslo issues two types of coin: aluminum (denoted A) and bronze (denoted B). Marianne has $n$ aluminum coins and $n$ bronze coins arranged in a row in some arbitrary initial order. A chain is any subsequence of consecutive coins of the same type. Given a fixed positive integer $k \leq 2n$, Gilberty repeatedly performs the following operation: he identifies the longest chain containing the $k^{th}$ coin from the left and moves all coins in that chain to the left end of the row. For example, if $n=4$ and $k=4$, the process starting from the ordering $AABBBABA$ would be $AABBBABA \to BBBAAABA \to AAABBBBA \to BBBBAAAA \to ...$\\
	Find all pairs $(n,k)$ with $1 \leq k \leq 2n$ such that for every initial ordering, at some moment during the process, the leftmost $n$ coins will all be of the same type.
	\begin{solution} We claim that $(n,k)$ is such a pair if and only if it is contained in the following set defined as,
$$S=\bigg\{(n,k)\in\mathbb{Z}_{\ge 1}^2\bigg\lvert n\le k\le n+\bigg\lceil\cfrac{n}{2}\bigg\rceil\bigg\}$$Firstly note that for $k<n$ we can construct a counter sequence of coins which shows that the final state cannot be achieved;
$$\underbrace{A\ldots A}_{\text{$n-1$ $A$'s}}\underbrace{C\ldots C}_{\text{$n-1$ $C$'s}}AC.$$
	\indent Similarly we show that $k\le n+\lceil n/2\rceil$ by constructing the sequence which loops to itself every time it completes a cycle. Also, to keep things neat and clean denote the contiguous string of coins of the same metal $M$ of size $\#$ as $M^{\#}$, consider the sequence;
$$A^{\big\lfloor\frac{n}{2}\big\rfloor} C^{\big\lceil\frac{n}{2}\big\rceil} A^{\big\lceil\frac{n}{2}\big\rceil} C^{\big\lfloor\frac{n}{2}\big\rfloor}\to C^{\big\lfloor\frac{n}{2}\big\rfloor} A^{\big\lfloor\frac{n}{2}\big\rfloor} C^{\big\lceil\frac{n}{2}\big\rceil} A^{\big\lceil\frac{n}{2}\big\rceil}\to \cdots \to A^{\big\lfloor\frac{n}{2}\big\rfloor} C^{\big\lceil\frac{n}{2}\big\rceil} A^{\big\lceil\frac{n}{2}\big\rceil} C^{\big\lfloor\frac{n}{2}\big\rfloor}\to \cdots $$
	Now, we are only left with the case when $n\le k\le n+\lceil n/2\rceil$ and as per our claim this range of values of $k$ works. Assume that for some $k$ in this range, some sequences do not attain the favourable end state. And for this to happen, there has to be a stagnant point after which the size of any contiguous string of coins of the same metal does not increase at all if not it has to reach the end state.\\

	\indent We would like to characterize the sequences that have reached the stagnant point. Note that the right-most contiguous string of the same metal has to have at least $2n+1-k$ coins at each iteration, else, it will contradict the stagnant point. With this argument, we assert that size of every contiguous string of coins of the same metal has to be at least $2n+1-k$ and as $n\le k\le n+\lceil n/2\rceil$ we may certainly state that the stagnant point is the end state itself and we are done. $\qedwhite$
	\end{solution}
\end{problem}
	
\begin{problem}{4}{\href{https://artofproblemsolving.com/community/q1h1578516p32946994}{India P5, 2018}}
	There are $n\ge 3$ girls in a class sitting around a circular table, each having some apples with her. Every time the teacher notices a girl having more apples than both of her neighbours combined, the teacher takes away one apple from that girl and gives one apple each to her neighbours. Prove that, this process stops after a finite number of steps. (Assume that, the teacher has an abundant supply of apples.)
	\begin{solution} Let $\mathcal{G}$ denote the set of girls $g_i$ where $i=1,2,\ldots,n $ and let $\alpha_j(g)$ denote the number of apples with $g$ after $j^{\text{th}}$ iteration for $j=0,1,\ldots,n$.\\
	Something cool pops up when we study the extremal object $\displaystyle\max_{g\in\mathcal{G}}\alpha_j{(g)}$.

	\begin{lemma}{1}{Monovariant Property}
		$\displaystyle\max_{g\in\mathcal{G}}\alpha_{j}{(g)}\ge \displaystyle\max_{g\in\mathcal{G}}\alpha_{j+1}{(g)}$.
	\end{lemma}
	By the above result we have an upper bound for the total number of apples in the class i.e,
$$\sum_{g\in\mathcal{G}}\alpha_{j}(g)\le |\mathcal{G}|\max_{g\in\mathcal{G}}\alpha_{0}{(g)}.$$But the total number of apples after each iteration seems to be increasing by $1$ which forces this process to terminate after a finite number of iterations.$\qedwhite$
	\end{solution}
\end{problem}
