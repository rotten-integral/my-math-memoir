\chapter{Geometry}
\indent It's very hard to sort geometry problems as each problem involves ideas from several different kinds. So it's very likely that I will not sort out these problems as I have for other disciplines.

\section{Assorted Geometry}
\subsection{Problems}
%Problem 1
\begin{problem}{1}{\href{https://artofproblemsolving.com/community/c6h3054399p27522960}{EGMO Day 2 P6, 2023}}
	Let $ABC$ be a triangle with circumcircle $\Omega$. Let $S_b$ and $S_c$ respectively denote the midpoints of the arcs $AC$ and $AB$ that do not contain the third vertex. Let $N_a$ denote the midpoint of arc $BAC$ (the arc $BC$ including $A$). Let $I$ be the incenter of $ABC$. Let $\omega_b$ be the circle that is tangent to $AB$ and internally tangent to $\Omega$ at $S_b$, and let $\omega_c$ be the circle that is tangent to $AC$ and internally tangent to $\Omega$ at $S_c$. Show that the line $IN_a$, and the lines through the intersections of $\omega_b$ and $\omega_c$, meet on $\Omega$.
\end{problem}

%Problem 2
\begin{problem}{2}{\href{https://artofproblemsolving.com/community/c6h488511p2737425}{IMO P5, 2012}} 
	Let $ABC$ be a triangle with $\angle BCA=90^{\circ}$, and let $D$ be the foot of the altitude from $C$. Let $X$ be a point in the interior of the segment $CD$. Let $K$ be the point on the segment $AX$ such that $BK=BC$. Similarly, let $L$ be the point on the segment $BX$ such that $AL=AC$. Let $M$ be the point of intersection of $AL$ and $BK$.\\
	Show that $MK=ML$.
\end{problem}

%Problem 3
\begin{problem}{3}{\href{https://artofproblemsolving.com/community/u571929h1640635p35544553}{Balkan P1, 2018}} 
	A quadrilateral $ABCD$ is inscribed in a circle $k$ where $AB$ $>$ $CD$,and $AB$ is not paralel to $CD$.Point $M$ is the intersection of diagonals $AC$ and $BD$, and the perpendicular from $M$ to $AB$ intersects the segment $AB$ at a point $E$.If $EM$ bisects the angle $CED$ prove that $AB$ is diameter of $k$.
\end{problem}

%Problem 4
\begin{problem}{4}{\href{https://artofproblemsolving.com/community/q2h1920217p35604706}{Int'l Festival of Young Mathematicians 2019}} 
	The perpendicular bisector of $AB$ of an acute $\Delta ABC$ intersects $BC$ and the continuation of $AC$ in points $P$ and $Q$ respectively. $M$ and $N$ are the middle points of side $AB$ and segment $PQ$ respectively. If the lines $AB$ and $CN$ intersect in point $D$, prove that $\Delta ABC$ and $\Delta DCM$ have a common orthocenter.
\end{problem}

%Problem 5
\begin{problem}{5}{\href{https://artofproblemsolving.com/community/q2h84559p35507946}{USAMO P6, 2006}} 
	Let $ABCD$ be a quadrilateral, and let $E$ and $F$ be points on sides $AD$ and $BC$, respectively, such that $\frac{AE}{ED} = \frac{BF}{FC}$. Ray $FE$ meets rays $BA$ and $CD$ at $S$ and $T$, respectively. Prove that the circumcircles of triangles $SAE$, $SBF$, $TCF$, and $TDE$ pass through a common point.
\end{problem}

%Problem 6
\begin{problem}{6}{\href{https://artofproblemsolving.com/community/q2h1819300p35507923}{EGMO P4, 2019}} 
	Let $ABC$ be a triangle with incentre $I$. The circle through $B$ tangent to $AI$ at $I$ meets side $AB$ again at $P$. The circle through $C$ tangent to $AI$ at $I$ meets side $AC$ again at $Q$. Prove that $PQ$ is tangent to the incircle of $ABC$.
\end{problem}

%Problem 7
\begin{problem}{7}{\href{https://artofproblemsolving.com/community/q2h3548106p35507758}{EGMO P4, 2025}} 
	Let $ABC$ be an acute triangle with incentre $I$ and $AB \neq AC$. Let lines $BI$ and $CI$ intersect the circumcircle of $ABC$ at $P \neq B$ and $Q \neq C$, respectively. Consider points $R$ and $S$ such that $AQRB$ and $ACSP$ are parallelograms (with $AQ \parallel RB, AB \parallel QR, AC \parallel SP$, and $AP \parallel CS$). Let $T$ be the point of intersection of lines $RB$ and $SC$. Prove that points $R, S, T$, and $I$ are concyclic.
\end{problem}

%Problem 8
\begin{problem}{8}{\href{https://artofproblemsolving.com/community/q2h130813p35493023}{IMO Shortlist G3, 2006}} 
	Let $ ABCDE$ be a convex pentagon such that
\[ \angle BAC = \angle CAD = \angle DAE\qquad \text{and}\qquad \angle ABC = \angle ACD = \angle ADE. \]The diagonals $BD$ and $CE$ meet at $P$. Prove that the line $AP$ bisects the side $CD$.
\end{problem}

%Problem 9	
\begin{problem}{9}{\href{https://artofproblemsolving.com/community/q2h449171p35492771}{Turkey TST 1998}} 
	In a triangle $ABC$, the circle through $C$ touching $AB$ at $A$ and the circle through $B$ touching $AC$ at $A$ have different radii and meet again at $D$. Let $E$ be the point on the ray $AB$ such that $AB = BE$. The circle through $A$, $D$, $E$ intersect the ray $CA$ again at $F$ . Prove that $AF = AC$.
\end{problem}

%Problem 10
\begin{problem}{10}{\href{https://artofproblemsolving.com/community/q2h224628p35453908}{USA TST Day 3-P7, 2008}} 
	Let $ ABC$ be a triangle with $ G$ as its centroid. Let $ P$ be a variable point on segment $ BC$. Points $ Q$ and $ R$ lie on sides $ AC$ and $ AB$ respectively, such that $ PQ \parallel AB$ and $ PR \parallel AC$. Prove that, as $ P$ varies along segment $ BC$, the circumcircle of triangle $ AQR$ passes through a fixed point $ X$ such that $ \angle BAG = \angle CAX$.
\end{problem}

%Problem 11
\begin{problem}{11}{\href{https://artofproblemsolving.com/community/q2h1876785p35453764}{India TST Day1-P1, 2019}} 
	In an acute angled triangle $ABC$ with $AB < AC$, let $I$ denote the incenter and $M$ the midpoint of side $BC$. The line through $A$ perpendicular to $AI$ intersects the tangent from $M$ to the incircle (different from line $BC$) at a point $P$> Show that $AI$ is tangent to the circumcircle of triangle $MIP$.
\end{problem}

%Problem 12
\begin{problem}{12}{\href{https://artofproblemsolving.com/community/q1h404355p35105708}{USAJMO P5, 2011}} 
	Points $A,B,C,D,E$ lie on a circle $\omega$ and point $P$ lies outside the circle. The given points are such that 
	\begin{itemize}
		\item [(i)] lines $PB$ and $PD$ are tangent to $\omega$,
		\item [(ii)] $P, A, C$ are collinear, and 
		\item[(iii)] $DE \parallel AC$.
	\end{itemize}
	Prove that $BE$ bisects $AC$.
\end{problem}

%Problem 13
\begin{problem}{13}{\href{https://artofproblemsolving.com/community/c6h39093p243438}{IMO Shortlist G8, 2004}} 
	Given a cyclic quadrilateral $ABCD$, let $M$ be the midpoint of the side $CD$, and let $N$ be a point on the circumcircle of triangle $ABM$. Assume that the point $N$ is different from the point $M$ and satisfies $\frac{AN}{BN}=\frac{AM}{BM}$. Prove that the points $E$, $F$, $N$ are collinear, where $E=AC\cap BD$ and $F=BC\cap DA$.
\end{problem}

%Problem 14
\begin{problem}{14}{\href{https://artofproblemsolving.com/community/c6h3100572p28033718}{ELMO Shortlist G1, 2023}} 
	Let $ABCDE$ be a cyclic pentagon. Let $P$ be a variable point on the interior of segment $AB$ such that $PA\ne PB$. The circumcircles of $\triangle PAE$ and $\triangle PBC$ meet again at $Q$. Let $R$ be the circumcenter of $\triangle DPQ$. Show that as $P$ varies, $R$ lies on a fixed line.\\
	\emph{Note.} The original problem had the condition that $ABCDE$ is regular but in this solution we will see that it is not necessary as cyclicity is sufficient.
\end{probelm}

%Problem 15
\begin{problem}{15}{Crux Mathematicorum 5044} 
	Given a circle $\Gamma$ with center $O$ and a chord $AB$, let $X$ be the midpoint of the larger arc $AB$, and $C$ be an arbitrary point of that arc. Define $K$ to be the point where the bisector of $\angle ACB$ intersects the tangent to $\Gamma$ at $B$, while $M$ is the intersection of $AC$ and $BX$. Prove that the line $MK$ contains the midpoint of $AB$.
\end{problem}


\newpage
\subsection{Solutions}
%Problem 1
\begin{problem}{1}{\href{https://artofproblemsolving.com/community/c6h3054399p27522960}{EGMO Day 2 P6, 2023}}
	Let $ABC$ be a triangle with circumcircle $\Omega$. Let $S_b$ and $S_c$ respectively denote the midpoints of the arcs $AC$ and $AB$ that do not contain the third vertex. Let $N_a$ denote the midpoint of arc $BAC$ (the arc $BC$ including $A$). Let $I$ be the incenter of $ABC$. Let $\omega_b$ be the circle that is tangent to $AB$ and internally tangent to $\Omega$ at $S_b$, and let $\omega_c$ be the circle that is tangent to $AC$ and internally tangent to $\Omega$ at $S_c$. Show that the line $IN_a$, and the lines through the intersections of $\omega_b$ and $\omega_c$, meet on $\Omega$.
	\begin{solution} Let $S_a$ be the midpoint of arc $BC$ not containing $A$ (it is also the antipode of $N_a$ in $\Omega$). Define $X\neq N_a$ to be the meeting point of $IN_a$ and $\Omega$.\\
	\par The problem simply asks us to show that $X$ lies on the radical axis of $\omega_b$ and $\omega_c$.
	
	\begin{claim}
	Point $A$ lies on the radical axis of $\omega_b$ and $\omega_c$.
	\end{claim}
	\begin{proof} It suffices to show that $\text{Pow}_{\omega_b}A=\text{Pow}_{\omega_c}A$.
	\par Shooting star lemma tells us that $\omega_b$ is tangent to $AB$ at a point say $M$ such that it passes through $\overline{S_bS_c}$ and by symmetry, one can argue the same for $\omega_c$ as well (define a point $N$ on $\overline{AC}$ such that $AN$ is tangent to $\omega_c$).By Incentre-Excentre lemma, it follows that $\odot (S_b, |\overline{S_bA}|)$ and $\odot (S_c, |\overline{S_cA}|)$ pass through $C$ and $B$ respectively. Now, as $AI$ is the radical axis of $\odot (S_b, |\overline{S_bA}|)$ and $\odot (S_c, |\overline{S_cA}|)$, it follows that $\overline{AI}\perp\overline{S_bS_c}$. We finish off by sending line $BC$ to line $S_BS_c$ by perspectivity through $A$. As $AN_a\parallel S_bS_c$, it follows that,
	$$-1=(BC; AI\cap BC,AN_a\cap BC)\overset{A}{=}(MN; AI\cap S_bS_c,\infty_{S_bS_c}).$$
	So $|AM|=|AN|$ as desired.
	\end{proof}
	
	Define $K$ to be the radical centre of circles $\Omega$, $\omega_b$ and $\omega_c$ (also $K$ passes through the radical axis of $\omega_b$ and $\omega_c$). And note, $KS_b$ and $KS_c$ are tangent to $\omega_b$ and $\omega_c$. So it suffices to show, $XA$ is the $X-$ symmedian of $\triangle XS_bS_c$.
	\begin{claim}
		The cyclic quadrilateral $AS_cXS_b$ is harmonic i.e,
		$$-1=(AX;S_cS_b)_{\Omega}.$$
	\end{claim}
	\begin{proof} Recall that $N_a$ is the antipode of $S_a$ in $\Omega$, meaning $N_aBS_aC$ is harmonic. Consider sending $\Omega$ to itself by perspectivity through $I$ to get,
	$$-1=(N_aS_a;BC)_{\Omega}\overset{I}{=}(XA;S_bS_c)_{\Omega},$$
	as desired.
	\end{proof}

	\noindent This indeed solves the problem and we are done.$\qedwhite$
	\begin{remark}[title=Comment.$\hspace{1mm}$]
	First I thought of performing a Bary-bash with reference triangle $ABC$ and then showing $\text{Pow}_{\omega_b}A=\text{Pow}_{\omega_c}A$ as PoP has a nice form in Barycentric coordinates. But the main problem is, determining the equations of $\omega_b$ and $\omega_c$ (maybe that's why you didn't spot a single Bary-bash solution in this thread). I even tried complex bashing it, but here the main problem is, how you translate the PoP characterization.
Does this problem really justify it's placement? Not really as it's mostly based on well known configs and requires little to no revelatory observations.
	\end{remark}
	\end{solution}
	\end{problem}

%Problem 2
\begin{problem}{2}{\href{https://artofproblemsolving.com/community/c6h488511p2737425}{IMO P5, 2012}} 
	Let $ABC$ be a triangle with $\angle BCA=90^{\circ}$, and let $D$ be the foot of the altitude from $C$. Let $X$ be a point in the interior of the segment $CD$. Let $K$ be the point on the segment $AX$ such that $BK=BC$. Similarly, let $L$ be the point on the segment $BX$ such that $AL=AC$. Let $M$ be the point of intersection of $AL$ and $BK$.\\
	Show that $MK=ML$.
	\begin{solution} Naturally we construct $\omega_A$, $\omega_B$ centred at $A$ and $B$ with radii $|AC|$ and $|BC|$ respectively. We note that $CD$ is the radical axis of $\omega_A$ and $\omega_B$. Let the two circles meet again at point say $E\neq C$. Moreover, these two circles are orthogonal.\\
	\indent We now make a slightly unnatural move that is to extend $AX$ to meet $\omega_B$ again at $P\neq K$. Similarly let $BX$ meet $\omega_A$ again at $Q\neq L$. Consider the following key claim.

		\begin{numclaim}{1}
			The cyclic quadrilateral $CKEP$ is harmonic. In other words,
			$$-1=(CE; KP)_{\omega_B}\overset{C}{=}(AX; KP).$$
		\end{numclaim}
	\begin{proof}
		Simply follows from the fact that $AC$ and $AE$ are tangent $\omega_B$ at $C$ and $E$ respectively, meaning, $A$ is the pole of line $CE$ with regards to $\omega_B$.
	\end{proof}

	\noindent Notice as $X$ lies on the radical $\omega_A$ and $\omega_B$, it follows, 
	$$\text{Pow}_{\omega_A}(X)=\text{Pow}_{\omega_B}(X)\quad\text{which in turn implies, $LPQK$ is cyclic.}$$
	Aha, we get that $LPQK$ is harmonic and furthermore, $AL$ is tangent to $(BPQK)$ at $L$ and by symmetry one can argue that $BK$ is tangent to $(BPQK)$ at $K$ as well. And we are done. $\qedwhite$\\
	
	\begin{remark}
	Here is the motivation behind extending $AX$ and $BX$ to meet $\omega_B$ and $\omega_A$ at $P$ and $Q$  respecitively: I had initially thought of performing a bary bash (it is more than good that I abandoned it), inorder to compute $K$, you'll have to first rule out the the case of $P$ as it lies on the same cevian $AX$ by consider a quadratic, then Vieta blah blah blah...
	\end{remark}
	\end{solution}
\end{problem}
	
%Problem 3
\begin{problem}{3}{\href{https://artofproblemsolving.com/community/u571929h1640635p35544553}{Balkan P1, 2018}} 
	A quadrilateral $ABCD$ is inscribed in a circle $k$ where $AB$ $>$ $CD$,and $AB$ is not paralel to $CD$.Point $M$ is the intersection of diagonals $AC$ and $BD$, and the perpendicular from $M$ to $AB$ intersects the segment $AB$ at a point $E$.If $EM$ bisects the angle $CED$ prove that $AB$ is diameter of $k$.
	\begin{numsolution}{1} We present a rather natural Bary Bash solution.\\
Notice, how showing $M$ is the incentre of $\triangle ECD$ suffices. This naturally suggests us to set $\triangle ECD$ as the reference. Let $E=(1:0:0)$, $C=(0:0:1)$ and $D=(0:0:1)$. Consider the labelling, $a=CD$, $b=DE$ and $c=EC$. As $EM$ is the angle bisector of $\angle CED$, we have $M=(t_M:b:c)$ for some $t_M\in\mathbb{R}$. And as $AB$ is the external angle bisector of $\angle ECD$, it follows that,
$$A=(t_A:-b:c)\quad\text{and}\quad B=(t_B:-b:c)$$for some $t_A, t_B\in\mathbb{R}$. Using the fact that points $C$, $M$ and $A$ lie on the same line, we have
$$\text{det}\begin{bmatrix} 0 & 1 & 0\\ t_M & b & c\\ t_A & -b & c \end{bmatrix}=-t_Mc+ct_A=0.$$So, $t_A=t_M$ and similarly using the fact that $D$, $M$ and $B$ lie on the same line it follows $t_B=-t_M$. Now, let's consider the equation of $k$ i.e,
$$k:-a^2yz-b^2zx-c^2xy+(x+y+z)ux=0\quad\text{for some $u\in\mathbb{R}$.}$$As $A$ lies on $k$, it follows that,
$$k(A)=a^2bc-b^2ct_M+c^2t_Mb+(t_M-b+c)ut_M=0\quad\implies\quad u=\frac{-a^2bc+b^2ct_M-c^2t_Mb}{t_M^2-t_Mb+ct_M}.$$The above can be treated as a quadratic in $t_M$ i.e,
$$ut_M^2+t_M\underbrace{(-ub+cu-b^2c+c^2b)}_{=0}+a^2bc=0.$$As $B=(-t_M:-b:c)$ lies on $k$, it must be that even $-t_M$ satisfies the above equation. So by Vieta's formula, the coefficient of $t_M$ is $0$. Which then yields us,
$$u=-bc.$$So the quadratic simply boils down to the following,
$$-bct_M^2+a^2bc=0\quad\implies\quad t_M=\pm a.$$But as $M$ lies in the interior of $\triangle ECD$, $t_M>0$ meaning $t_M=a$ and hence $M=(a:b:c)$ which is precisely the coordinates of the incentre of $\triangle ECD$ and we are done. $\qedwhite$
	\end{numsolution}

	\begin{numsolution}{2}
	This setup immediately induces complete quadrilateral $ABCD$, so naturally we define $N=AD\cap BC$ and $P=AB\cap CD$ (not to mention neither $N$ nor $P$ are points at infinity). Consider the following key result.

	\begin{claim}
		Points $E$, $M$ and $N$ lie on the same line.
	\end{claim}
	\begin{proof} Define $Q=EM\cap DC$ and define the phantom point $Q'=NM\cap PC$. We wish to show that $Q$ and $Q'$ coincide. Consider the perspectivity through $N$ sending line $PB$ to line $PC$, then by \bluehighlight{Ceva-Menelaus} we have,
$$(A,B; NM\cap PB, P)\overset{N}{=}(D, C; Q', P)=-1.$$But we know that $(D,C; Q,P)=-1$, so $Q'$ coincides with $Q$.
	\end{proof}
	
	\indent Now, by \bluehighlihgt{Brokard's theorem}, it follows that the circumcentre $O$ of $ABCD$ is the orthocentre of $\triangle MNP$ or equivalently, $M$ is the orthocentre of $\triangle ONP$ which means $OP\perp MN$. But by claim $1$, we know that $AB\perp MN$ meaning $O$ must lie on $\overline{AB}$ and we are done. $\qedwhite$
	\end{numsolution}
\end{problem}
	
	%Problem 4
\begin{problem}{4}{\href{https://artofproblemsolving.com/community/q2h1920217p35604706}{Int'l Festival of Young Mathematicians 2019}} 
	The perpendicular bisector of $AB$ of an acute $\Delta ABC$ intersects $BC$ and the continuation of $AC$ in points $P$ and $Q$ respectively. $M$ and $N$ are the middle points of side $AB$ and segment $PQ$ respectively. If the lines $AB$ and $CN$ intersect in point $D$, prove that $\Delta ABC$ and $\Delta DCM$ have a common orthocenter.
	\begin{solution} Let $\triangle XYZ$ be the orthic triangle of $\triangle ABC$ with $X$, $Y$ and $Z$ lying on $\overline{BC}$, $\overline{CA}$ and $\overline{AB}$ respectively. Define $H$ to be the orthocentre of $\triangle ABC$. Notice that it suffices to show $MH\perp CD$.

	\begin{numclaim}{1} Points $X$, $Y$ and $D$ lie on the same line.
		\end{numclaim}
	\begin{proof} Consider the perspectivity through $C$ sending line $PQ$ to line $AB$ by which we have,
$$(PQ; N\infty_{PQ})\overset{C}{=}(BA; DZ)=-1.$$And on the other hand by Ceva-Menelaus we have that $(BA; XY\cap AB Z)=-1$. Thus, $XY\cap AB=D$ and the result follows.\\
	\end{proof}
	
	Now to finish off, we invoke Brokard's theorem on the complete cyclic quadrilateral $ABXY$. As $M$ is the circumcentre of $(ABXY)$ it follows that $MH\perp CD$ as desired and we are done. $\qedwhite$
	\end{solution}
	\end{problem}
	
% Problem 5
\begin{problem}{5}{\href{https://artofproblemsolving.com/community/q2h84559p35507946}{USAMO P6, 2006}} 
	Let $ABCD$ be a quadrilateral, and let $E$ and $F$ be points on sides $AD$ and $BC$, respectively, such that $\frac{AE}{ED} = \frac{BF}{FC}$. Ray $FE$ meets rays $BA$ and $CD$ at $S$ and $T$, respectively. Prove that the circumcircles of triangles $SAE$, $SBF$, $TCF$, and $TDE$ pass through a common point.
	\begin{solution} Let $M$ be the Miquel point of $ABFE$. It suffices to show that $M$ is also the Miquel point of $EFCD$.\\
We know that there exists a spiral similarity centred at $M$ such that $A\mapsto E$ and $E\mapsto F$. As $E$ and $F$ divide $\overline{AD}$ and $\overline{BC}$ respectively in the same ratio, it follows that $M$ is the spiral centre with $E\mapsto D$ and $F\mapsto C$(recall that spiral similarity is just a composition of homothety and rotation). Aha! by uniqueness of spiral centre, $M$ is indeed the Miquel point of $EFCD$ and we are done. $\qedwhite$
	\end{solution}
\end{problem}

%Problem 6
\begin{problem}{6}{\href{https://artofproblemsolving.com/community/q2h1819300p35507923}{EGMO P4, 2019}} 
	Let $ABC$ be a triangle with incentre $I$. The circle through $B$ tangent to $AI$ at $I$ meets side $AB$ again at $P$. The circle through $C$ tangent to $AI$ at $I$ meets side $AC$ again at $Q$. Prove that $PQ$ is tangent to the incircle of $ABC$.
	\begin{solution} Let the second tangent to the incircle through $P$ other than $PB$ touch it at point $T$. It suffices to show that $Q=PT\cap AC$ which is equivalent to showing that $\angle BPT=\angle BPQ$.
	
	\begin{numclaim}{1} $BCQP$ is cyclic.
	\end{numclaim}
	
	\begin{proof} Begin by noticing that $AI$ is the radical axis of $(BIP)$ and $(CIQ)$. It follows that
$$\text{Pow}_{(BIP)}A=\text{Pow}_{(CIQ)}A\quad\text{which implies $BCQP$ is cyclic.}$$
	\end{proof}
	
	\noindent Now, consider the following angle chase.
	\begin{align*} \angle BPT &=2\angle BPI && \text{($\angle IPT=\angle BPI$)} \\ &=2(180^{\circ}-\angle AIB) && \text{($AI$ is tangent to $(BIP)$ at $I$)}\\ &=2\left(\frac{\angle BAC}{2}+\frac{\angle CBA}{2}\right) && \text{(angle sum in $\triangle AIB$)}\\ &=\angle BAC+\angle CBA\\ &=180^{\circ}-\angle ACB.
	\end{align*}
	\par So, the points $P$, $B$, $C$ and $PT\cap AC$ are concyclic but we have already seen that $Q$ lies on $(PBC)$ which means $Q=PT\cap AC$ and we are done. $\qedwhite$
	\end{solution}
\end{problem}

%Problem 7
\begin{problem}{7}{\href{https://artofproblemsolving.com/community/q2h3548106p35507758}{EGMO P4, 2025}} 
	Let $ABC$ be an acute triangle with incentre $I$ and $AB \neq AC$. Let lines $BI$ and $CI$ intersect the circumcircle of $ABC$ at $P \neq B$ and $Q \neq C$, respectively. Consider points $R$ and $S$ such that $AQRB$ and $ACSP$ are parallelograms (with $AQ \parallel RB, AB \parallel QR, AC \parallel SP$, and $AP \parallel CS$). Let $T$ be the point of intersection of lines $RB$ and $SC$. Prove that points $R, S, T$, and $I$ are concyclic.
	\begin{solution} We shall first show that $T$ lies on $(BIC)$ and then proceed to show, there exists a spiral similarity centred at $I$ with $B\mapsto R$ and $C\mapsto S$. Then by uniqueness of spiral centre, we would end up showing that $R$, $S$, $T$, and $I$ are concyclic.
	\begin{numclaim}{1} 
		Points $B$, $C$, $T$ and $I$ are concyclic.
	\end{numm claim}
	\begin{proof} It suffices to show $\measuredangle BTC=\measuredangle BIC$. Consider the following angle chase.
	\begin{align*}
	\measuredangle ACT &=180^{\circ}-\angle SCA &&\text{($T$, $C$ and $S$ are collinear)}\\ &=\angle APS &&\text{($ACSP$ is a parallelogram)}\\ &=(180^{\circ}-\angle CBA)+\angle CPS &&\text{($\angle APS=\angle APC+\angle CPS$)}\\ &=(180^{\circ}-\angle CBA)+\frac{\angle CBA}{2}&&\\ &=180^{\circ}-\frac{\angle CBA}{2}.&&
	\end{align*}
	\par So, $\measuredangle ACT=\frac{\angle CBA}{2}$ and by symmetry, we end up with $\measuredangle TBA=\frac{\angle ACB}{2}$. Now, as $\measuredangle ACT+\angle TCB=\angle ACB$ it follows that $\angle TCB=\frac{\angle CBA}{2}$. Again by symmetry we get $\angle CBT=\frac{\angle ACB}{2}$ and thereby the result follows.
	\end{proof}
	
	\begin{numclaim}
		$\triangle IBR\overset{+}{\sim}\triangle ICS$.
	\end{numclaim}
	\begin{proof} We begin by showing that $\frac{IB}{BR}=\frac{IC}{CS}$. Consider the following length chase.
\begin{align*} \frac{IB}{BR}&=\frac{IB}{QA} && \text{($BR=QA$ as $AQRB$ is a parallelogram)}\\ &=\frac{IB}{QB} && \text{($Q$ is the midpoint of arc $BQA$)}\\ &=\frac{IC}{CS}.&& \text{($\triangle IBQ\sim \triangle TCP$)} \end{align*}We are just left to show that $\measuredangle ICS=\measuredangle IBR$ which follows from a simple angle chase.
\begin{align*} \measuredangle ICS&=180^{\circ}-\measuredangle TCI &&\text{(points $T$, $C$ and $S$ are collinear)}\\ &=180^{\circ}-\measuredangle TBI && \text{($BCTI$ is cyclic)}\\ &=\measuredangle IBR. && \text{(points $T$, $B$ and $R$ are collinear)} \end{align*}Hence the result.
	\end{proof}

	It is now evident that $I$ is the spiral centre sending $B$ to $C$ and $R$ to $S$ which means $I$ is also the spiral centre sending $B$ to $R$ and $C$ to $S$ and we are done. $\qedwhite$
	\end{solution}
\end{problem}

%Problem 8
\begin{problem}{8}{\href{https://artofproblemsolving.com/community/q2h130813p35493023}{IMO Shortlist G3, 2006}} 
	Let $ ABCDE$ be a convex pentagon such that
\[ \angle BAC = \angle CAD = \angle DAE\qquad \text{and}\qquad \angle ABC = \angle ACD = \angle ADE. \]The diagonals $BD$ and $CE$ meet at $P$. Prove that the line $AP$ bisects the side $CD$.
	\begin{solution} As $(ABC)$ and $(ADE)$ are tangent to $\overline{CD}$ at $C$ and $D$ respectively, it suffices to show that $AP$ is the radical axis of these two circles as it would instantly imply that $\text{Pow}_{(ABC)}(AP\cap CD)=\text{Pow}_{(ADE)}(AP\cap CD)$.

	\begin{numclaim}{1} $P$ lies on $(ABC)$ and $(ADE)$.
	\end{numclaim}
	\begin{proof} Observe that $A$ is the spiral centre sending $B$ to $D$ and $C$ to $E$. And as $P=\overline{BD}\cap \overline{CE}$, it follows that $P$ lies on $(ABC)$ and $(ADE)$ by uniqueness of spiral centre.
	\end{proof}

	By the above claim, it is clear that $AP$ is the radical axis and we are done. $\qedwhite$
	\end{solution}
\end{problem}

%Problem 9	
\begin{problem}{9}{\href{https://artofproblemsolving.com/community/q2h449171p35492771}{Turkey TST 1998}} In a triangle $ABC$, the circle through $C$ touching $AB$ at $A$ and the circle through $B$ touching $AC$ at $A$ have different radii and meet again at $D$. Let $E$ be the point on the ray $AB$ such that $AB = BE$. The circle through $A$, $D$, $E$ intersect the ray $CA$ again at $F$ . Prove that $AF = AC$.
	\begin{numsolution}{1} We invoke Barycentric Coordinates. Let $ABC$ be the reference triangle with $A=(1:0:0)$, $B=(0:1:0)$ and $C=(0:0:1)$. Also we follow the usual convention $a=BC$, $b=CA$ and $c=AB$. Define $F'$ as a point on the ray $CA$ with $F'A=AC$ i.e, $F'=(2:0:-1)$. Our goal is to show that $F'$ satisfies the Barycentric equation of $(ADE)$.
Note that $D$ is simply the $A-$ dumpty point, so using the fact that it is also the midpoint of $A-$symmedian chord of $(ABC)$ we get $D=(2S_A: b^2:c^2)$. Lastly $E=(-1:2:0)$ indeed. As $(ADE)$ passes through $A$, the equation is given by
$$\Omega: -a^2yz-b^2zx-c^2xy+(x+y+z)(vy+wz)=0\quad\text{for some $v,w\in\mathbb{R}$.}$$Evaluating $\Omega$ at point $E$,we get,
$$\Omega(E)=2c^2+2v=0\quad\text{which implies $v=-c^2$.}$$Now, similarly evaluating $\Omega$ at $D$, we get,
\begin{align*} \Omega(D)&=-a^2b^2c^2-4S_Ab^2c^2+(2b^2+2c^2-a^2)(-c^2b^2+wc^2)\\ &=-a^2b^2-4S_Ab^2+(2b^2+2c^2-a^2)(-b^2+w)\\ &=0 \quad\text{meaning, $w=\frac{4S_Ab^2+a^2b^2}{2b^2+2c^2-a^2}+b^2$.} \end{align*}Let's evaluate $\Omega$ at $F'$ and we have,
\begin{align*} \Omega(F')&=2b^2+\left(\frac{-4S_Ab^2-a^2b^2}{2b^2+2c^2-a^2}-b^2\right)\\ &=b^2-\underbrace{\frac{4S_Ab^2-a^2b^2}{2b^2+2c^2-a^2}}_{=b^2}\\ &=0. \end{align*}Therefore $F'$ coincides with $F$ and we are done. $\qedwhite$
	\end{numsolution}
	
	\begin{numsolution}{2} Define $F'$ to be the point on ray $CA$ with $F'A=AC$ and our goal is to show that $F'$ coincides with $F$. It is evident that $D$ is the spiral centre sending $B$ to $A$ and $A$ to $C$. Call the spiral similarity $\sigma$. The key result is the following.

	\begin{proposition}{1} Let $X$ be a point on line $AB$ and $Y$ be the point on line $AC$ with $X\xmapsto{\sigma}Y$. Then it follows that,
$$\frac{AX}{XB}=\cfrac{CY}{YA}.$$
	\end{proposition}
	\noindent Proof is redundant.

	\par From the above result, we can say that $E\xmapsto{\sigma} F'$ or in other words $\triangle EDA\sim \triangle F'DC$. On the other hand, $F$ lies on both $(ADE)$ and $(FDC)$ which implies $D$ is the spiral centre with $E\mapsto F$ and $A\mapsto C$. Aha, this means that both spiral similarities are identical and hence $F'$ coincides with $F$ and we are done. $\qedwhite$
	\end{numsolution}
\end{problem}
	
%Problem 10
\begin{problem}{10}{\href{https://artofproblemsolving.com/community/q2h224628p35453908}{USA TST Day 3-P7, 2008}} 
	Let $ ABC$ be a triangle with $ G$ as its centroid. Let $ P$ be a variable point on segment $ BC$. Points $ Q$ and $ R$ lie on sides $ AC$ and $ AB$ respectively, such that $ PQ \parallel AB$ and $ PR \parallel AC$. Prove that, as $ P$ varies along segment $ BC$, the circumcircle of triangle $ AQR$ passes through a fixed point $ X$ such that $ \angle BAG = \angle CAX$.
	\begin{solution} We shall use barycentric coordinates. Let $ABC$ be the reference triangle with $A=(1:0:0)$, $B=(0:1:0)$ and $C=(0:0:1)$. Parametrize $P$ as $(0:u:v)$ for $u,v\in\mathbb{R}$ with $u+v=a$. Then we blindly proceed to compute the coordinates of $Q, R$ using Thale's theorem by which we end with,
$$Q=(u:0:v)\quad\text{and}\quad R=(v:u:0).$$As we now have the coordinates of $A,Q$ and $R$, we can simply find the equation of $(AQR)$ and we shall refer it to as $\Gamma$. The equation is given by,
$$\Gamma: -a^2yz -b^2zx-c^2xy+(x+y+z)(v'y+w'z)=0\quad\text{for some $v', w'\in\mathbb{R}$.}$$Putting the coordinates of $Q$ and $R$ in the above equation tells us what these $v'$ and $w'$ are.
\begin{align*} \Gamma(Q)&=-b^2uv+a(w'u)=0 \quad\text{so, $w'=\frac{b^2u}{a}$.}\\ \Gamma(R)&=-c^2vu+a(v'v)=0 \quad\text{so, $v'=\frac{b^2u}{a}$.} \end{align*}So the refined form of $\Gamma$ is given by,
$$\Gamma (x:y:z)=-a^2yz-b^2zx-c^2xy+(x+y+z)\left(\frac{b^2u}{a}x+\frac{c^2v}{a}y\right)=0.$$As $AX$ is isogonal to $AG$, we can parametrize $X$ as $(t:b^2:c^2)$ for some $t\in\mathbb{R}$ (using the fact that symmedian point $K=(a^2:b^2:c^2)$ lies on $AX$). We then proceed to show that $t$ is fixed as we vary the point $P$.
\begin{align*} \Gamma(X)&= -a^2b^2c^2-b^2c^2t-c^2b^2t+(t+b^2+c^2)\left(\frac{c^2b^2}{a}v+\frac{b^2c^2}{a}u\right)\\ &=-a^2b^2c^2-2tb^2c^2+t\frac{b^2c^2}{a}(u+v)+(b^2+c^2)(u+v)\frac{b^2c^2}{a}\\ &=-t-a^2+b^2+c^2\\ &=0. \end{align*}Hence $t=2S_A$ meaning $\Gamma$ always passes through $X=(2S_A:b^2:c^2)$ and we are done. $\qedwhite$

	\begin{remark} 
		$X$ is simply the $A-$dumpty point i.e, the midpoint of $A-$symmedian chord of $(ABC)$.
	\end{remark}
	\end{solution}
\end{problem}
	
%Problem 11
\begin{problem}{11}{\href{https://artofproblemsolving.com/community/q2h1876785p35453764}{India TST Day1-P1, 2019}} 
	In an acute angled triangle $ABC$ with $AB < AC$, let $I$ denote the incenter and $M$ the midpoint of side $BC$. The line through $A$ perpendicular to $AI$ intersects the tangent from $M$ to the incircle (different from line $BC$) at a point $P$> Show that $AI$ is tangent to the circumcircle of triangle $MIP$.
	\begin{solution} It suffices to show that $\measuredangle IMP=\measuredangle AIP$. Define $D$ as the point of contact of the incircle and line $BC$ and let $D$ be the antipode of $D$ on the incircle. Let $E$ be the reflection of $D$ in $M$ and say line $MP$ is tangent to the incircle at point $X$. The key result is the following.

	\begin{numclaim}{1}
		The points $A, D', X$ and $E$ all lie on the same line.
	\end{numclaim}
	\begin{proof} We only need to show that $X$ lies on $AD'$ as it is well known that $A, D'$ and $E$ are collinear. Consider a homothety $\mathcal{H}_D(2)$. It is evident that
$$I\xmapsto{\mathcal{H}}D'\quad\text{and}\quad M\xmapsto{\mathcal{H}} E.$$As $X$ is the reflection of $D$ in the line $MI$, it follows that $\mathcal{H}$ sends the foot of $D$ on $MI$ to $X$ and hence the claim.
	\end{proof}

	By the virtue of $\mathcal{H}$, it follows that $\overline{IM}\parallel \overline{AE}$. We then have,
\begin{align*} \measuredangle AIP &= \measuredangle AXP && \text{$A, I, X, P$ are concyclic,}\\ &= \measuredangle EXM && \text{vertically opposite angles,}\\ &= \measuredangle IMX && \text{alternate interior angles,}\\ &= \measuredangle IMP && \text{$M, X, P$ are collinear.} \end{align*}Which is exactly what we wished to show and we are done. $\qedwhite$
	\end{solution}
\end{problem}
	
%Problem 12
\begin{problem}{12}{\href{https://artofproblemsolving.com/community/q1h404355p35105708}{USAJMO P5, 2011}} 
	Points $A,B,C,D,E$ lie on a circle $\omega$ and point $P$ lies outside the circle. The given points are such that 
	\begin{itemize}
		\item [(i)] lines $PB$ and $PD$ are tangent to $\omega$,
		\item [(ii)] $P, A, C$ are collinear, and 
		\item[(iii)] $DE \parallel AC$.
	\end{itemize}
	Prove that $BE$ bisects $AC$.
	\begin{solution} As $ABCD$ is harmonic it suffices to show that $BD$ and $BE$ are isogonal conjugates in $\triangle BCD$ i.e, $\angle DBC=\angle ABE$ (this is because $BD$ is the $B$-symmedian of $\triangle BCA$). Clearly it follows that,
$$\angle DBC=\angle DEC=\angle ACE=\angle ABE.$$Hence $BE$ bisects $AC$. $\qedwhite$
	\end{solution}
\end{problem}

%Problem 13
\begin{problem}{13}{\href{https://artofproblemsolving.com/community/c6h39093p243438}{IMO Shortlist G8, 2004}} 
	Given a cyclic quadrilateral $ABCD$, let $M$ be the midpoint of the side $CD$, and let $N$ be a point on the circumcircle of triangle $ABM$. Assume that the point $N$ is different from the point $M$ and satisfies $\frac{AN}{BN}=\frac{AM}{BM}$. Prove that the points $E$, $F$, $N$ are collinear, where $E=AC\cap BD$ and $F=BC\cap DA$.
	\begin{solution} Define $H=CD\cap EF$ and $G=AB\cap CD$. As $(DC; HG)=-1$ and $M$ being the midpoint of $\overline{CD}$ it follows that 
	$$GH\cdot GM=GD\cdot GC=GA\cdot GB.$$
	This implies $H$ lies on $(ABM)$. Now by considering the perspectivity sending line $AB$ to $(ABM)$ through $H$, we have,
	$$-1=(AB; G, EF\cap AB)\overset{H}{=}(AB; M, EF\cap (ABM)).$$
	But we also know that $(AB; MN)=-1$ meaning $EF\cap (ABM)=N$ and we are done. $\qedwhite$
	\end{solution}
\end{problem}

%Problem 14
\begin{problem}{14}{\href{https://artofproblemsolving.com/community/c6h3100572p28033718}{ELMO Shortlist G1, 2023}} 
	Let $ABCDE$ be a cyclic pentagon. Let $P$ be a variable point on the interior of segment $AB$ such that $PA\ne PB$. The circumcircles of $\triangle PAE$ and $\triangle PBC$ meet again at $Q$. Let $R$ be the circumcenter of $\triangle DPQ$. Show that as $P$ varies, $R$ lies on a fixed line.\\
	\emph{Note.} The original problem had the condition that $ABCDE$ is regular but in this solution we will see that it is not necessary as cyclicity is sufficient.
	\begin{solution} Note that $X:=SD\cap (ABCDE)$ lies on $(PDQ)$ as $AE\cap BC$ is the radical centre of $(PDQ)$, $(ABCDE)$ and $(PBCQ)$. So $R$ must always lie on the perpendicular bisector of $XD$, and we are done.$\qedwhite$
	\end{solution}
\end{problem}

%Problem 15
\begin{problem}{15}{Crux Mathematicorum 5044} Given a circle $\Gamma$ with center $O$ and a chord $AB$, let $X$ be the midpoint of the larger arc $AB$, and $C$ be an arbitrary point of that arc. Define $K$ to be the point where the bisector of $\angle ACB$ intersects the tangent to $\Gamma$ at $B$, while $M$ is the intersection of $AC$ and $BX$. Prove that the line $MK$ contains the midpoint of $AB$.
	\begin{solution} We present a \bluehighlight{Projective solution}. Define $P$ as the midpoint of arc $AB$ of $\Gamma$ not containing $X$ and define $N$ to be the midpoint of line segment $AB$. Call the tangent to $\Gamma$ at $B$ as $\ell$. We wish to prove that $M, N$ and $K$ are collinear. The key idea is to define two projective maps $\varphi_1, \varphi_2 : \widehat{AXB} \to \ell$ with
	
	\begin{itemize}
		\item[{1.}] $\varphi_1 : C \mapsto CP \mapsto CP \cap \ell \overset{\text{def}}{=} K_1$, 
		\item[{2.}] $\varphi_2 : C \mapsto AC \cap BX = M \mapsto MN \cap \ell \overset{\text{def}}{=} K_2$,
	\end{itemize} 
	\noindent then showing these two maps are identical, finishes the problem as it would imply $K_1 \equiv K_2 \equiv K$ hence $M, N$ and $K$ are collinear.
	\par We begin by showing $\varphi_1, \varphi_2$ are projective. The map with $C \mapsto CP$ 
preserves cross ratios thus is projective and similarly the map with $CP \mapsto K_1$ 
is projective as well. Now, as composition of two projective maps is also projective, 
it follows that $\varphi_1$ is projective. Following similar lines, one can conclude 
that $\varphi_2$ is projective as well. The following result is the reason for 
emphasizing on the `projective' nature of these maps.
	
	\begin{lemma}{1} Let $f,g : C_1 \to C_2$ be two projective maps that coincide 
at at least three distinct points then it follows that $f$ and $g$ are identical. 
Where $C_1,C_2$ could be conic sections, pencil of lines or a single line.
	\end{lemma}
	\begin{proof} Let $A_1,A_2,A_3$ be three distinct points on $C_1$ such that 
$f,g$ coincide on all three points. Then, as for any $B \in C_2 \setminus \{A_1,A_2,A_3\}$ 
there exists a unique point $X$ on $C_2$ such that
\[
(f(A_1), f(A_2); f(A_3), X) = (g(A_1), g(A_2); g(A_3), X) = (A_1, A_2; A_3, B).
\]
	Thus, it follows that $f(B) = g(B) = X$ as desired.
	\end{proof}

	\noindent It follows from \textit{lemma 1} that it suffices to check for three distinct cases for $C$:
	\begin{itemize}
    	\item Consider $C \equiv B$. Quite evidently $C \xmapsto{\varphi_1} B$ and $C \xmapsto{\varphi_2} B$.
    
    	\item Consider $C \equiv X$. As $X \equiv M$ it follows that $C \xmapsto{\varphi_1} XP \cap \ell$ and 
    $C \xmapsto{\varphi_2} XP \cap \ell$.
    
    	\item Lastly we consider a slightly trickier case i.e.\ $C \equiv A$. We begin by noting that in the 
    limiting case, the line $CA$ is simply the tangent to $\Gamma$ at $A$. Let $Y$ be the meeting point of 
    the tangent to $\Gamma$ at $A$ and $\ell$. It is easy to see that $APBX$ is a harmonic quadrilateral and hence
    \[
    (X,P;A,B) = (X,P;N,Y) = -1.
    \]
    
    	As $\angle XBP = 90^\circ$, it follows that $BP$ is the angle bisector of $\angle ABY$ and by symmetry 
    	$P$ is the incenter of $\triangle ABY$. Define $K = AP \cap BY$ and let $M' = NK \cap AY$. 
    	It is well known that $(A,Y; BP \cap AY, M') = -1$ and as we have already seen that $BX$ is the external 
    	angle bisector of $\angle ABY$, it follows that $(A,Y; BP \cap AY, M) = -1$. And this implies $M' \equiv M$, 
    	so $\varphi_1, \varphi_2$ coincide once again for the third time.
	\end{itemize}

	\noindent So $\varphi_1, \varphi_2$ are indeed identical and we are done. $\qedwhite$
	
	\begin{remark}[title=Comment.$\hspace{1mm}$] Note that the restricted location of $C$ in the larger arc $AB$ is not required as the result follows otherwise. The method we used is an advanced projective technique which mainly shows up in the world of high school olympiads and is often referred to as `Method of Moving Points' or `MMP' for short.
	\end{remark}
	\end{solution}
\end{problem}
